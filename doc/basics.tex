\section{Basics}
\subsection{A very basic setup}

Every information on the the material and structures you want to cut out must be stored in an INI file. In the listing below is an example of a very basic setup defining a sphere made up of Natriumchloride:
\lstinputlisting{srcexamples/basic.ini}
Every input file producing any output consists at least of two sections. The geometry section, containing everything needed to specify the crystals structure, and at least one body section (opened by \lstinline{[sphere: somename]} in this case), defining the body to be cut out.

\subsection{Usage}
The simplest valid command is \lstinline{nanocut INFILE}. INFILE must be a valid INI file like the one specified in the example above. The result will be written to the standard output in xyz format. Options can be added to the commandline at arbitrary positions. Possible Options are:
\begin{description}
 \item{-w FILE} This writes the output to the file specified by FILE. In case FILE doesn't exist it will be created, otherwise the existing file will be overwritten without further questions.
 \item{-a FILE} This merges the result with the content of an existing INI file specified by FILE. FILE must exist.
 \item{-h} Prints helptext.
 \item{--help} Prints helptext.
\end{description}
Usually the command looks like this: \lstinline{nanocut basic.ini -w basic.xyz}. With basic.ini containing the configuration specified in the listing above the output looks like this:
TODO INCLUDE IMAGE OF basic.xyz>>>>>>>
