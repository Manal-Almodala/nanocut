\subsubsection{Periodic convex prism}
The periodic convex prism is an infinitely long cylinder with a convex polygon as base area. The prism is defined by its lateral planes. A plane can be defined using it's miller indices or it's normal vector. The body's section is opened by \lstinline{[convex_polyhedron: NAME]}.

If a plane is not parallel to the periodicity axis the plane is rotated until it is paralell. The rotation axis is orthogonal to the normal vector of the plane and the periodicity axis and intersects with the origin. 

\paragraph{Parameters}
\begin{description}
 \item{\lstinline{planes_miller}} Miller indices of each plane (except those defined using normal vectors) followed by its distance from the origin.
 \item{\lstinline{planes_normal}} Vector orthogonal to each plane (except those defined using miller indices) followed by its distance from the origin. The vectors do not need to be normalized.
\end{description} 


\paragraph{Example}\

\lstinputlisting{srcexamples/periodic_1D_convex_prism.ini}
\ \\\includegraphics[width=0.6\textwidth]{srcexamples/periodic_1D_convex_prism.png}